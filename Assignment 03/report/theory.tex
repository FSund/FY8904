%!TEX root = report.tex
\subsection*{The periodic surface Rayleigh equation}
We will solve numerically the \emph{periodic surface Rayleigh equation}
\begin{equation}
    \sum_{\vec K_{\|}'} \hat I\del[2]{-\alpha_0 \del[1]{K_{\|}', \omega} \big\vert \vec K_{\|} - \vec K_{\|}'} M\del[1]{\vec K_{\|} \big\vert \vec K_{\|}'} r\del[1]{\vec K_{\|}' \big\vert \vec k_{\|}} 
    = -\hat I \del[2]{\alpha_0 \del[1]{k_{\|}, \omega} \big\vert \vec K_{\|} - \vec k_{\|}} N\del[1]{\vec K_{\|} \big\vert \vec k_{\|}},
    \label{eq:rayleigh}
\end{equation}
or
\begin{equation}
    \sum_{\vec K_{\|}'} \hat I\del[2]{-\alpha_0 \del[1]{K_{\|}', \omega} \big\vert \vec G_{\|} - \vec G_{\|}'} M\del[1]{\vec K_{\|} \big\vert \vec K_{\|}'} r\del[1]{\vec K_{\|}' \big\vert \vec k_{\|}} 
    = -\hat I \del[2]{\alpha_0 \del[1]{k_{\|}, \omega} \big\vert \vec G_{\|}} N\del[1]{\vec K_{\|} \big\vert \vec k_{\|}},
\end{equation}
where the lateral wave vectors $\vec K_{\|}$ and $\vec K_{\|}$ are defined as
\begin{align}
    &\vec K_{\|} = \vec k_{\|} + \vec G_{\|} &\vec K_{\|}' = \vec k_{\|} + \vec G_{\|}',
\end{align}
and $\vec G_{\|}$ are the lattice sites of the reciprocal lattice of the doubly periodic surface profile $\xi(\vec x)$, given by
\begin{align}
    &\vec G_{\|} (\vec h) = h_1 \vec b_1 + h_2 \vec b_2, \qquad h_i \in \mathbb{Z}.
\end{align}
We will use a square lattice with translation vectors $\vec a_1 = a \hat {\vec x}_1$ and $\vec a_2 = a\hat{\vec x}_2$ which means that the reciprocal lattice vectors are $\vec b_1 = (2\pi/a)\hat{\vec x}_1$ and $\vec b=(2\pi/a)\hat{\vec x}_2$, and
\begin{align}
    &\vec G_{\|} (\vec h) = h_1 \frac{2\pi}{a} \hat{\vec x}_1 + h_2 \frac{2\pi}{a} \hat{\vec x}_2, \qquad h_i \in \mathbb{Z}.
\end{align}
The wave vector $\vec k$ represents the incident wave, and is written in the form
\begin{equation}
    \vec k = \vec k_\| \pm \alpha_0(k_\|, \omega)\hat {\vec x}_3
\end{equation}
with
\begin{equation}
    \alpha_0(k_\|, \omega) =
    \begin{cases}
        \sqrt{\frac{\omega^2}{c^2} - k_\|^2} &k_\|^2 < \frac{\omega^2}{c^2} \\
        i\sqrt{k_\|^2 - \frac{\omega^2}{c^2}} &k_\|^2 \geq \frac{\omega^2}{c^2}
    \end{cases}.
\end{equation}
The wavelength of the incident beam is denoted by $\lambda$, and is related to the angular frequency $\omega$ via $\omega/c = 2\pi/\lambda$. From geometry considerations it can be shown that
\begin{equation}
    \vec k_\| = \frac{\omega}{c}\sin\theta_0 \del[1]{\cos\phi_0, \sin\phi_0, 0}.
\end{equation}

The set of solutions $\cbr[1]{r\del[1]{\vec K_{\|}'\big\vert \vec k_{\|}}}$ of \cref{eq:rayleigh} describes the reflection of an incident scalar wave of lateral wave vector $\vec_\|$ that is scattered by a periodic surface $\xi(\vec x_\|)$ into reflected waves characterized by the wave vector $\vec K_\|'$.

The $\hat I$-integrals are defined in the next section.

To be able to solve \cref{eq:rayleigh} we limit the values of
\begin{equation}
    \vec K_{\|}'(\vec h) = \vec k + \vec G_\|'(\vec h)
\end{equation}
by limiting the components of $\vec h = (h_1, h_2)$ to
\begin{equation}
    h_i \in \sbr{-H, H} ~~(h_i \in \mathbb{Z}),
\end{equation}
where $H$ is a positive integer. We then have a finite set of $N = n^2 = (2H+1)^2$ unknown scattering amplitudes $r(\vec K_\|' \big\vert \vec k_\|)$. We then let $\vec K_\|$ take the same values as $\vec K_\|'$, which gives us $N$ different variants of \cref{eq:rayleigh}. We can then express \cref{eq:rayleigh} as a linear system of $N$ equations and $N$ unknowns, $\vec A \vec x = \vec b$, where
\begin{equation}
    \vec A =
    \begin{pmatrix}
        A_{1,1} & A_{1,2} & \dots  & A_{1,N} \\
        A_{2,1} & A_{2,2} & \dots  & A_{2,N} \\
        \vdots  & \vdots  & \ddots & \vdots  \\
        A_{N,1} & A_{N,2} & \dots  & A_{N,N} \\
    \end{pmatrix}
    \label{eq:theMatrix}
\end{equation}
where $A_{i, j}$ is the pre-factor before $r$ in the sum in \cref{eq:rayleigh},
\begin{equation}
    A_{i, j} = \hat I\del[2]{-\alpha_0 \del[1]{K_{\|}'^j, \omega} \big\vert \vec K_{\|}^i - \vec K_{\|}'^j} M\del[1]{\vec K_{\|}^i \big\vert \vec K_{\|}'^j},
\end{equation}
and
\begin{align}
    \vec K_\|^{i} = \vec K_\| (\vec h_i) \\
    \vec K_\|'^{j} = \vec K_\|' (\vec h_j)
\end{align}
% ($\vec h_i$ is specified below).
Further we have
\begin{equation}
    \vec x =
    \begin{pmatrix}
        r\del[2]{\vec K_\|'^{1} \vert \vec k_\|} \\
        r\del[2]{\vec K_\|'^{2} \vert \vec k_\|} \\
        \hdots \\
        r\del[2]{\vec K_\|'^{N-1} \vert \vec k_\|} \\
        r\del[2]{\vec K_\|'^{N} \vert \vec k_\|}
    \end{pmatrix}
\end{equation}
and
% \begin{align*}
%     &\cbr{\vec h_i} = \vec h_1, \vec h_2, \dots, \vec h_n \\
%     &\phantom{\cbr{\vec h_i}{}} = \del{h_1, h_1}, \del{h_1, h_2}, \dots, \del{h_1, h_{n-1}}, \del{h_1, h_n}, \\
%     &\phantom{\cbr{\vec h_i} = {}} \del{h_2, h_1}, \del{h_2, h_2}, \dots, \del{h_2, h_{n-1}}, \del{h_2, h_n}, \\
%     &\hspace{5cm} \vdots \\
%     &\phantom{\cbr{\vec h_i} = {}} \del{h_{n-1}, h_1}, \del{h_{n-1}, h_2}, \dots, \del{h_{n-1}, h_{n-1}}, \del{h_{n-1}, h_n}, \\
%     &\phantom{\cbr{\vec h_i} = {}} \del{h_n, h_1}, \del{h_n, h_2}, \dots, \del{h_n, h_{n-1}}, \del{h_n, h_n} \\
% \end{align*}
\begin{align*}
    \cbr{\vec h_i} = \vec h_1, \vec h_2, \dots, \vec h_n =
    \begin{matrix}
        \del{h_1, h_1}, &\del{h_1, h_2}, &\dots &\del{h_1, h_{n-1}}, &\del{h_1, h_n}, \\
        \del{h_2, h_1}, &\del{h_2, h_2}, &\dots &\del{h_2, h_{n-1}}, &\del{h_2, h_n}, \\
        \vdots &\vdots &\ddots &\vdots &\vdots  \\
        \del{h_{n-1}, h_1}, &\del{h_{n-1}, h_2}, &\dots &\del{h_{n-1}, h_{n-1}}, &\del{h_{n-1}, h_n}, \\
        \del{h_n, h_1}, &\del{h_n, h_2}, &\dots &\del{h_n, h_{n-1}}, &\del{h_n, h_n}.
    \end{matrix}
\end{align*}
($\cbr{\vec h_i}$ is not not a matrix, but is represented in a matrix form above to more easily make the connection to the matrix in \cref{eq:theMatrix}). In practice this is implemented as
\begin{equation}
    \vec h_ i = (h_j, h_k) ~~\text{where}~~ j = i \sslash n ~~\text{and}~~ k = i \bmod n,
\end{equation}
where $\sslash$ is integer division and $\mathrm{mod}$ is the \emph{modulo} operator. This allows us to loop over the linear index $i$ in the code.

\subsection*{The $\hat I$-integral}
For a \emph{doubly periodic cosine profile} of period $a$ and amplitude $\xi_0$ we can calculate the $\hat I$-integral in closed form as \cite{maradudin2018features}
\begin{equation}
    \hat I\del[1]{\gamma\vert \vec G_\|(\vec h)} = (-i)^{h_1} \mathrm{J}_{h_1}\del{\frac{\gamma\xi_0}{2}} (-i)^{h_2} \mathrm{J}_{h_2}\del{\frac{\gamma\xi_0}{2}},
\end{equation}
where $\mathrm{J}_n(\cdot)$ is the Bessel function of first kind and order $n$. The Bessel functions are evaluated via the \texttt{SciPy} function \texttt{scipy.special.jv} with the argument \texttt{order=n}.

For a truncated cone surface profile it can be shown that \cite{maradudin2018features}
\begin{align}
    \hat I\del[1]{\gamma\vert \vec G_\|(\vec h)} = {}
    &\delta_{\vec G_\|, \vec 0} + 2\pi \frac{\rho_t^2}{a^2} \sbr[2]{\exp\del{-i\gamma\xi_0}}\frac{J_1\del[1]{G_\| \rho_t}}{G_\| \rho_t} \\
    & + 2\pi\frac{\rho_b - \rho_t}{a^2} \sum_{n=1}^{\infty} \frac{\del{-i\gamma\xi_0}^n}{n!}
    \int_0^1 \dif u_\| \sbr[1]{\rho_b - \del[0]{\rho_b - \rho_t}u_\|} J_0 \del[2]{G_\| \sbr[1]{\rho_b - \del[0]{\rho_b - \rho_t}u_\|}} u_\|^n,
    \label{eq:truncatedCone}
\end{align}
where $\delta$ is the Kronecker-delta, and a change of variable has been performed
\begin{equation}
    u_\| = \frac{\rho_b - x_\|}{\rho_b - \rho_t}.
\end{equation}

Finally, for a truncated cosine surface profile we have \cite{maradudin2018features}
\begin{align}
    \hat I\del[1]{\gamma\vert \vec G_\|(\vec h)} = 
    \delta_{\vec G_\|, \vec 0} + 
    \frac{2\pi}{a^2} \sum_{n=1}^{\infty} \frac{\del{-i\gamma}^n}{n!}
    \int_0^{\rho_0} \dif x_\| x_\| J_0 \del[1]{G_\| x_\|}\xi^n\del[0]{x_\|}
    \label{eq:truncatedCosine}
\end{align}

The integrals in \cref{eq:truncatedCone,eq:truncatedCosine} were evaluated numerically, and enough terms were included in the sum for it to converge (typically 5-10 terms were needed).

\subsection*{Nondimensionalizing}
% We now have all the components we need to solve \cref{eq:rayleigh}, but first we will non-dimensionalize the equations. Using the wavelength as the length scale $x_0 = \lambda$ we 
We will be using the wavelength as the length scale $x_0 = \lambda$ when numerically solving the equations.