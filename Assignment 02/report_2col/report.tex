%!TEX options = -shell-escape

\documentclass[a4paper,twocolumn]{article}
\usepackage[utf8]{inputenc} % Løser problem med å skrive andre enn engelske bokstaver f.eks æ,ø,å.
\usepackage[T1]{fontenc} % Støtter koding av forskjellige fonter.
\usepackage{amsmath}
\usepackage{amssymb} % for set of integers Z etc. 
\usepackage{bm}
\usepackage{enumitem}
\usepackage{soulutf8}
\usepackage[normalem]{ulem} % \sout and \xout for strikethrough/cancel text
\usepackage{layouts} % so we can do \printinunitsof{in}\prntlen{\textwidth}
\usepackage{mathtools}
\usepackage{commath2, commath2-additions}
\usepackage[parfill]{parskip}
\usepackage[theorems]{tcolorbox}  % load theorems for tcboxmath support
\usepackage[cm]{fullpage}
\usepackage[%
    backend=biber, % biblatex is the package, biber is the (default). The alternative is backend=bibtex, but biber should be better.
    % sorting=none, % "sorting=none" means "sorting=citeorder" (in order of citation)
    sorting=nty, % name, title, year
    style=numeric,
    giveninits=true, % only want first ("given") name as initials -- doesn't work with authoryear
    maxbibnames=99, % show all names in bibliography
]{biblatex}
\addbibresource{bibliography.bib}

% required load order: (float - fix \listoflistings) - hyperref - minted
\usepackage[section]{placeins}
\usepackage{hyperref}

%% minted %%
\usepackage[newfloat]{minted}
\usepackage{xcolor}
\usemintedstyle{colorful}
\definecolor{codebg}{rgb}{0.95,0.95,0.95}
% \definecolor{codehl}{HTML}{FDF6E3}
\definecolor{codehl}{rgb}{0.90,0.90,0.90}

%% CPP %%
\newminted[cppcode]{cpp}{ % use \begin{cppcode}
    mathescape,
    bgcolor = codebg,
    fontsize = \footnotesize,
    breaklines,
}
\newminted[plaincppcode]{cpp}{ % use \begin{cppplaincode}
    mathescape,
    fontsize = \footnotesize,
    breaklines,
}
\newmintinline[cppinline]{cpp}{breaklines} % use \cppinline
\newmint[cppmint]{cpp}{breaklines}
\newmintedfile[cppfile]{cpp}{ % use \cppfile[<options>]{<filename>}
    mathescape,
    bgcolor = codebg,
    fontsize = \footnotesize,
    breaklines,
}
%% %% %% %%
%% PYTHON %%
\newmintinline[pyinline]{python}{breaklines} % use \pyinline
%% %% %% %%

\usepackage[capitalise]{cleveref}
\usepackage{graphicx}
\graphicspath{{../figs/}}
\usepackage{subcaption}
\usepackage{todonotes}
\usepackage[font=small,labelfont=bf]{caption}

%% commands %%
\newcommand{\cpp}{\texttt{C++}}
\newcommand{\python}{\texttt{Python}}
\newcommand{\cppeleven}{\texttt{C++11}}

%% "task x.x enumerate list" %%
% \newlist{tasks}{enumerate}{1}
% \setenumerate[tasks]{wide, labelwidth=!, labelindent=0pt, listparindent=0pt, label=\textbf{Task \thesection.\arabic*}}

\setlength{\belowcaptionskip}{0.0pt} % 0.0pt
\setlength{\abovecaptionskip}{8.0pt} % 10.0pt

\title{FY8904 Assignment 2}
\date{Spring 2019}
\author{Filip Sund}

\begin{document}
\maketitle

\begin{abstract}
    In this report we have studied the particle in box and the double well problem, using computational physics, comparing the results to analytical/exact results where possible. We have found that a method using expansion in eigenstates reproduces analytical results well, and simulated quantum tunnelling between the two lowest eigenstates of the double well system. \hl{We have set up a time dependent double well system  Results. Discussion. Conclusions.}
\end{abstract}
% \section*{Notes on code}
% Most of my code is \cpp\, and make extensive use of the Armadillo \cpp\ matrix library\footnote{See \url{http://arma.sourceforge.net/docs.html} for documentation.}. 

% \begin{itemize}
%     \item \cppinline{Mat} is the base matrix type in Armadillo
%     \item \cppinline{Col} is the base column vector type in Armadillo
%     \item \mintinline{c++}{mat} is typedef for \mintinline{c++}{Mat<double>}
%     \item \mintinline{c++}{vec} is typedef for \mintinline{c++}{Col<double>}
%     \item \mintinline{c++}{umat} is typedef for \mintinline{c++}{Mat<uword>}
%     \item \mintinline{c++}{imat} is typedef for \mintinline{c++}{Mat<sword>}
%     \item \mintinline{c++}{uvec} is typedef for \mintinline{c++}{Col<uword>}
%     \item \mintinline{c++}{ivec} is typedef for \mintinline{c++}{Col<sword>}
%     \item \mintinline{c++}{uword} is typedef for an \emph{unsigned} integer type
%     \item \mintinline{c++}{sword} is typedef for an \emph{signed} integer type
% \end{itemize}
% The minimum width for \mintinline{c++}{uword} and \mintinline{c++}{sword} is 64 bits on 64-bit platforms when using \cppeleven\ and newer standards, and 32 bits when using older \cpp\ standards.

\section*{Introduction}

textwidth: \printinunitsof{in}\prntlen{\textwidth}
linewidth: \printinunitsof{in}\prntlen{\linewidth}

\section*{Theory}
%!TEX root = report.tex
\subsection*{The periodic surface Rayleigh equation}
We will solve numerically the \emph{periodic surface Rayleigh equation}
\begin{equation}
    \sum_{\vec K_{\|}'} \hat I\del[2]{-\alpha_0 \del[1]{K_{\|}', \omega} \big\vert \vec K_{\|} - \vec K_{\|}'} M\del[1]{\vec K_{\|} \big\vert \vec K_{\|}'} r\del[1]{\vec K_{\|}' \big\vert \vec k_{\|}} 
    = -\hat I \del[2]{\alpha_0 \del[1]{k_{\|}, \omega} \big\vert \vec K_{\|} - \vec k_{\|}} N\del[1]{\vec K_{\|} \big\vert \vec k_{\|}},
    \label{eq:rayleigh}
\end{equation}
where the lateral wave vectors $\vec K_{\|}$ and $\vec K_{\|}$ are defined as
\begin{align}
    &\vec K_{\|} = \vec k_{\|} + \vec G_{\|} &\vec K_{\|}' = \vec k_{\|} + \vec G_{\|}',
\end{align}
and $\vec G_{\|}$ are the lattice sites of the reciprocal lattice of the doubly periodic surface profile $\xi(\vec x)$, given by
\begin{align}
    &\vec G_{\|} (\vec h) = h_1 \vec b_1 + h_2 \vec b_2, \qquad h_i \in \mathbb{Z}.
\end{align}
We will use a square lattice with translation vectors $\vec a_1 = a \hat {\vec x}_1$ and $\vec a_2 = a\hat{\vec x}_2$ which means that the reciprocal lattice vectors are $\vec b_1 = (2\pi/a)\hat{\vec x}_1$ and $\vec b=(2\pi/a)\hat{\vec x}_2$, and
\begin{align}
    &\vec G_{\|} (\vec h) = h_1 \frac{2\pi}{a} \hat{\vec x}_1 + h_2 \frac{2\pi}{a} \hat{\vec x}_2, \qquad h_i \in \mathbb{Z}.
\end{align}
The wave vector $\vec k$ represents the incident wave, and is written in the form
\begin{equation}
    \vec k = \vec k_\| \pm \alpha_0(k_\|, \omega)\hat {\vec x}_3
\end{equation}
with
\begin{equation}
    \alpha_0(k_\|, \omega) =
    \begin{cases}
        \sqrt{\frac{\omega^2}{c^2} - k_\|^2} &k_\|^2 < \frac{\omega^2}{c^2} \\
        i\sqrt{k_\|^2 - \frac{\omega^2}{c^2}} &k_\|^2 \geq \frac{\omega^2}{c^2}
    \end{cases}.
\end{equation}
The wavelength of the incident beam is denoted by $\lambda$, and is related to the angular frequency $\omega$ via $\omega/c = 2\pi/\lambda$.

The set of solutions $\cbr[1]{r\del[1]{\vec K_{\|}'\big\vert \vec k_{\|}}}$ of \cref{eq:rayleigh} describes the reflection of an incident scalar wave of lateral wave vector $\vec_\|$ that is scattered by the periodic surface $\xi(\vec x_\|$ into reflected waves characterized by the wave vector $\vec K_\|'$.

The $\hat I$-integrals \hl{are defined elsewhere}.

To be able to solve \cref{eq:rayleigh} we limit the values of
\begin{equation}
    \vec K_{\|}'(\vec h) = \vec k + \vec G_\|'(\vec h)
\end{equation}
by limiting the components of $\vec h = (h_1, h_2)$ to
\begin{equation}
    h_i \in \sbr{-H, H} ~~(h_i \in \mathbb{Z}),
\end{equation}
where $H$ is a positive integer. We then have a finite set of $N = n^2 = (2H)^2$ unknown scattering amplitudes $r(\vec K_\|' \big\vert \vec k_\|)$. We then let $\vec K_\|$ take the same values as $\vec K_\|'$, which gives us $N$ different evaluations of \cref{eq:rayleigh}. We can then express \cref{eq:rayleigh} as a linear system of equations $\vec A \vec x = \vec b$, where
\begin{equation}
    \vec A =
    \begin{pmatrix}
        A_{1,1} & A_{1,2} & \dots  & A_{1,N} \\
        A_{2,1} & A_{2,2} & \dots  & A_{2,N} \\
        \vdots  & \vdots  & \ddots & \vdots  \\
        A_{N,1} & A_{N,2} & \dots  & A_{N,N} \\
    \end{pmatrix}
\end{equation}
where $A_{i, j}$ is the pre-factor before $r$ in the sum in \cref{eq:rayleigh},
\begin{equation}
    A_{i, j} = \hat I\del[2]{-\alpha_0 \del[1]{K_{\|}'^j, \omega} \big\vert \vec K_{\|}^i - \vec K_{\|}'^j} M\del[1]{\vec K_{\|}^i \big\vert \vec K_{\|}'^j},
\end{equation}
and
\begin{align}
    \vec K_\|^{i} = \vec K_\| (\vec h_i) \\
    \vec K_\|'^{j} = \vec K_\|' (\vec h_j)
\end{align}
% ($\vec h_i$ is specified below).
Further we have
\begin{equation}
    \vec x =
    \begin{pmatrix}
        r\del[2]{\vec K_\|'^{1} \vert \vec k_\|} \\
        r\del[2]{\vec K_\|'^{2} \vert \vec k_\|} \\
        \hdots \\
        r\del[2]{\vec K_\|'^{N-1} \vert \vec k_\|} \\
        r\del[2]{\vec K_\|'^{N} \vert \vec k_\|}
    \end{pmatrix}
\end{equation}
and
\begin{align*}
    &\cbr{\vec h_i} = \vec h_1, \vec h_2, \dots, \vec h_n \\
    &\phantom{\cbr{\vec h_i}{}} = \del{h_1, h_1}, \del{h_1, h_2}, \dots, \del{h_1, h_{n-1}}, \del{h_1, h_n}, \\
    &\phantom{\cbr{\vec h_i} = {}} \del{h_2, h_1}, \del{h_2, h_2}, \dots, \del{h_2, h_{n-1}}, \del{h_2, h_n}, \\
    &\hspace{5cm} \vdots \\
    &\phantom{\cbr{\vec h_i} = {}} \del{h_{n-1}, h_1}, \del{h_{n-1}, h_2}, \dots, \del{h_{n-1}, h_{n-1}}, \del{h_{n-1}, h_n}, \\
    &\phantom{\cbr{\vec h_i} = {}} \del{h_n, h_1}, \del{h_n, h_2}, \dots, \del{h_n, h_{n-1}}, \del{h_n, h_n} \\
\end{align*}
In practice this is implemented as
\begin{equation}
    \vec h_ i = (h_j, h_k) ~~\text{where}~~ j = i \sslash n ~~\text{and}~~ k = i \bmod n,
\end{equation}
where $\sslash$ is integer division and $\mathrm{mod}$ is the \emph{modulo} operator.

\subsection*{The $\hat I$-integral}
For a \emph{doubly periodic cosine profile} of period $a$ and amplitude $\xi_0$ we can calculate the $\hat I$-integral in closed form
\begin{equation}
    \hat I\del[1]{\gamma\vert \vec G_\|(\vec h)} = (-i)^{h_1} \mathrm{J}_{h_1}\del{\frac{\gamma\xi_0}{2}} (-i)^{h_2} \mathrm{J}_{h_2}\del{\frac{\gamma\xi_0}{2}},
\end{equation}
where $\mathrm{J}_n(\cdot)$ is the Bessel function of first kind and order $n$. The Bessel functions are evaluated via the \texttt{SciPy} function \texttt{scipy.special.jv} with the argument \texttt{order=n}.

\section*{Results and discussion}
%!TEX root = report.tex
\subsection*{Particle in a box}


\section*{Conclusion}

\printbibliography[title=References]

\end{document}