%!TEX root = report.tex
% \begin{itemize}
%     \item Give dimensionless form (and define $x_0$ and $t_0$) of time-independent Scrhoedinger equation
% \end{itemize}
\subsection*{The Schrödinger equation}
The one-dimensional Schrödinger equations in dimensionless form reads (time-dependent)
\begin{equation}
    i2mL^2\pd{\Psi}{t'} = \hat H \Psi
\end{equation}
and (time-independent)
\begin{equation}
    E_n \psi_n = \hat H \psi_n,
    \label{eq:tise}
\end{equation}
where $E_n$ is the energy of the state $\psi_n$.

We have chosen the non-dimensionalizing scales
\begin{align}
    \frac{t}{t'} = t_0 = \frac{2mL^2}{\hbar} ~~\text{and}~~ \frac{x}{x'} = x_0 = L,
\end{align}
where $L$ is the size of the domain (the width of the infinite well in our case). 


% The one-dimensional Schrödinger equations in dimensionless form reads (time-dependent)
% \begin{equation}
%     i\pdt{\Psi} = -\pdx[2]{\Psi}
% \end{equation}
% and (time-independent)
% \begin{equation}
%     \lambda \psi_n = -\pdx[2]{\psi_n},
%     \label{eq:tise}
% \end{equation}
% where we have chosen the non-dimensionalizing scales
% \begin{align}
%     t_0 = \frac{2mL^2}{\hbar} ~~\text{and}~~ x_0 = L
% \end{align}
% and replaced $t$ and $x$ with $t'=t/t_0$ and $x' = x/x_0$, dropping the primes in the process. $L$ is the size of our domain (the width of the potential well), and the relation between $\lambda_n$ and $E_n$ is
% \begin{equation}
%     \lambda_n = E_n \frac{2mL^2}{\hbar^2}.
% \end{equation}

% Given an initial condition $\Psi_0(x) = \Psi_(x, t=0)$ and a time-independent Hamiltonian $H$, the Schrödinger equation has a formal solution
% \begin{equation}
%     \Psi = \exp\del{\frac{}{}}
% \end{equation}

\subsection*{Particle in a box}
For the particle in a box problem the boundary conditions are
\begin{equation}
    \Psi(x'=0, t') = 0 ~~\text{and}~~ \Psi(x'=1, t') = 0,
\end{equation}
and the (dimensionless) Hamiltonian is 
\begin{equation}
    \hat H = -\pd[2]{}{{x'}} + \nu(x'),
\end{equation}
where $\nu(x'\in \sbr{0, 1}) = 0$ and $\nu(x') = \infty$ elsewhere. The exact solution of \cref{eq:tise} is
\begin{equation}
    \psi_n(x') = \sqrt{2}\sin\del{n\pi x'} ~~\text{for}~~ n=1,2,3,\dots
    \label{eq:exact_eigenfunctions}
\end{equation}
with eigenvalues $\lambda_n = (\pi n)^2$. The relation between $\lambda_n$ and $E_n$ is
\begin{equation}
    \lambda_n = E_n \frac{2mL^2}{\hbar^2}.
\end{equation}

\subsubsection*{Box with potential barrier/double well}
For a box with a potential barrier the potential is modified as follows
\begin{equation}
\nu(x') = 
\begin{cases*}
    0 & for $0 < x' < 1/3$ \\
    \nu_0 & for $1/3 < x' < 2/3$ \\
    0 & for $2/3 < x' < 1$ \\
    \infty & otherwise,
\end{cases*}
\end{equation}
where $\nu_0 = t_0 V_0/\hbar$ is a dimensionless measure of the strength of the potential barrier.

\subsubsection*{Periodic detuning of a two-level system}
Introducing a time-dependent potential
\begin{equation}
\nu(x') = 
\begin{cases*}
    0 & for $0 < x' < 1/3$ \\
    \nu_0 & for $1/3 < x' < 2/3$ \\
    \nu_r(t) & for $2/3 < x' < 1$ \\
    \infty & otherwise,
\end{cases*}
\end{equation}
we can force population transfers between the two lowest energy levels.

\subsection*{Expansion in eigenfunctions}
If we know the expansion of the initial condition $\Psi_0$ in the eigenfunctions $\psi_n$
\begin{equation}
    \Psi_0(x') = \sum_n \alpha_n \psi_n(x')
\end{equation}
then the Schrödinger equation gives us the time evolution of the system
\begin{equation}
    \Psi(x', t') = \sum_n \alpha_n \exp\del{-i\lambda_n t'}\psi_n(x').
    \label{eq:time_evolution_expansion}
\end{equation}
The coefficients $\alpha_n$ can be calculated using the inner product between the initial state $\Psi_0$ and the eigenstates $\psi_n$
\begin{equation}
    \alpha_n = \langle \psi_n, \Psi_0 \rangle = \int \psi_n^{*}(x')\Psi_0(x') \dif x'.
    \label{eq:coeffs}
\end{equation}

\subsection*{Finite difference time evolution}
If we have a time-dependent Hamiltonian 
%If the coefficients $\alpha_n$ in \cref{eq:coeffs} are hard to evaluate, %
the method of expansion in eigenfunctions can not be used. An alternative method for time evolution is finite difference methods. We then use the formal solution of the Schrödinger equation for two times separated by $\Delta t'$
\begin{equation}
    \Psi(x', t' + \Delta t') = \exp(-i\Delta t' \hat H)\Psi(x', t').
\end{equation}
One approximation to this equation is the forward Euler scheme
\begin{equation}
    \Psi(x', t'+\Delta t') = \sbr{1 - i\Delta t' \hat H} \Psi(x', t').
    \label{eq:euler_scheme}
\end{equation}
This scheme does not preserve probabilities, since the approximation to the time evolution operator $\exp(-i\Delta t' \hat H)$ is not unitary. A better choice is the Crank-Nicholson scheme
\begin{equation}
    \sbr{1 + \frac{i}{2}\Delta t' \hat H} \Psi(x', t'+\Delta t') = \sbr{1 - \frac{i}{2}\Delta t' \hat H} \Psi(x', t'+\Delta t').
    \label{eq:cn_scheme}
\end{equation}

% \subsection*{Particle in a box}
% \begin{itemize}
%     \item Give hamiltonian
%     \item Describe boundary conditions
%     \item Give exact eigenvalues and eigenfunctions (eq. (2.10) and $(\pi n)^2$)
%     \item Give formulas for expansion in eigenfunctions (eq. 2.11, 2.12 and 2.14)
% \end{itemize}

% \begin{itemize}
%     \item \sout{Give hamiltonian for double well (eq. 3.2)}
%     \item \sout{Define dimensionless potential $\nu_0 = t_0V_0/\hbar$?}
%     \item \sout{Give eq. 3.4}
%     \item \sout{Give Euler scheme for time-evolution, eq. 3.5}
%     \item \sout{Give Crank-Nicholson scheme for time-evolution, eq. 3.8}
% \end{itemize}

\subsection*{Numerical details}
We solve the Schrödinger equations using finite difference, using the forward difference for the first derivative
\begin{equation}
    \pd{f(x)}{x} \approx \frac{f(x + \Delta x) - f(x)}{\Delta x}
\end{equation}
and the central difference for the second derivative
\begin{equation}
    \pd[2]{f(x)}{x} \approx \frac{f(x + \Delta x) - 2f(x) + f(x - \Delta x)}{\Delta x^2}
    .
\end{equation}
This allows us to rewrite the Schrödinger equations as an eigenvalue problem $(\bm A -\lambda_n \bm I) \bm x = 0$ ($\bm I$ is the identity matrix), which is solved using the \texttt{numpy} function \texttt{numpy.linalg.eigh}.

When setting up the matrix $\bm A$ we don't include the boundaries $\Psi(x=0)$ and $\Psi(x=L)$, since these are known to always be equal to zero, due to the infinite potential. This gives wave functions that are exactly equation to the analytical solution (within machine precision), no matter which $\Delta x$ we choose, as can be seen for example in \cref{fig:box_solutions}.